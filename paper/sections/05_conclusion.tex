\section{Conclusion}
\label{sec:conclusion}
In the end, we summarize our findings concerning both hard cut and soft cut detection.

\subsection{Hard cut detection}
\label{sec:conclusion_hard_cut}

As we see in Table~\ref{tab:hard_cut_results}, the low precision of our approach is a problem.
We have discussed the main reason for that in Section~\ref{sec:hard_cut_visualization}.
Our approach just takes two frames into account and therefore cannot notice, whether the transition under consideration is part of a noisy cluster or a single peak (see Figure~\ref{fig:hard_cut_noise_visualization}).
One idea to overcome this "blindness" is to use some surrounding histogram differences as additional features.
The classifier is trained with a sliding window of histogram differences, resulting in specific patterns.
The classification task would then be a pattern matching task, where hard cuts are characterized by a single peak, soft cuts by a gradual curve and noise by random peaks.
Future work could also try to employ a neural net for this kind of pattern matching.
A different approach is to do post-processing, where certain highly unlikely events are filtered out.
Looking at the false positives in Figure~\ref{fig:hard_cut_noise_visualization} (red bars), one can see, that many of them occur directly after each other.
Those cut patterns are not realistic.
They might either be noise or a soft cut.

Furthermore, our approach is not applicable for low quality black and white videos (like the \emph{BG\_11362} video of the \emph{TrecVid 2007} data set).
The problem is, that there are strong brightness changes from one frame to another within one scene.
Therefore, we can hardly distiguish between cuts and non-cuts.
An idea to solve this problem is to calculate the brightness difference between the raw pixel values.
If the resulting difference image has homogenious color, the brightness difference between the two frames is the result of bad image quality.


\subsection{Soft cut detection}
\label{sec:conclusion_hard_cut}

What went wrong:
More data?

In summary.
We still think that the soft cut detection could be solved with deep learning.
However, the basic approach with using a recurrent neural network does not work.
More work and innovation is required in this area.

We think that the main problem is that the convolutional network cannot really learn good features, because there is nothing characteristic about a soft cut frame.
