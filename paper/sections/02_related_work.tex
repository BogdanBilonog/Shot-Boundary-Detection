\section{Related Work}
\label{sec:related_work}

\subsection{Shot Boundary Detection}

Many different techniques have been developed for shot boundary detection.
For hard cut detection, a baseline approach compares the change between two frames for each pixel, and builds the sum of these differences.
Then a threshold is selected: Every value over the threshold is considered a cut.
However, this approach is prone to error, when there are quick camera movements, or sudden color changes inside a scene.
Therefore, better approaches have been developed, which are shortly summarized in the next paragraphs.

\paragraph{Color Histograms}
This approach, first used by Zhang~et.al.~\cite{zhang1993automatic}, sorts the color values into a histogram with a certain bin size, before comparing the histograms.
Thus, this approach is less prone errors as the baseline approach.
It can better compensate for minor changes in the frames.
However, using histograms also means compressing the data, thereby throwing some information away.

\paragraph{Luminance Values}
Instead of working on the RGB data, one can also work in other color spaces.
Especially luminance values~\cite{petersohn2004fraunhofer} have shown good performance, as they emphasize the white and black difference.
Thus, they are less susceptible to false positives.

\paragraph{Edge Detection}
Another approach is to detect the edges in a frame and the subsequent frame.
If the edges differ fundamentally, this is a sign for a hard cut.
In ~\cite{ewerth2005university}, the authors use ``edge histograms of Sobel-filtered (vertically and horizontally) DC-frames''.

\paragraph{Other Techniques}
There are many more finegrained ideas, which can be used to refine the result.
One is motion compensation, used by Qu{\'s}enot~et.al.~\cite{qusenot2003clips}, to prevent false positives from camera movements.
Another one is the idea of adaptive thresholding~\cite{volkmer2004rmit}, which does not use one fixed threshold but rather adapts the threshold dynamically, based on the current sequence.

\paragraph{Machine Learning}
For finding exact thresholds or decision boundaries, many approaches employ machine learning, especially via support vector machines (SVM).
This works by extracing features like histogram bins, edges and then pass these to a machine learning algorithm.
SVMs are popular, because they are ``easy to use and provide a quick classification time after the initial training''~\cite{smeaton2010video}.


It is also possible to combine many of the approaches from above.
For finding soft cuts, the state of the art technique is to use a sliding window of frames.
The decision from many frame-to-frame comparisons can then be incorporated into a final soft cut prediction.

\subsection{Deep Neural Networks}
Deep neural networks were very successful in image classification~\cite{krizhevsky2012imagenet} and video classifaction~\cite{karpathy2014large} tasks in the last years.
In the ImageNet Large Scale Visual Recognition Challenge 2014 (ILSVRC2014)~\cite{imagenet}, almost all contestants were using neural networks.
This recent rise in popularity in the computer vision community has been fueled by two recent developments.
First, the rise in computing power, especially of GPUs, has enabled reasonable training times.
Second, with more and more data being collected, more training data is available, which was crucial for the success of these techniques.

Neural networks work by processing the raw input pixel values layer by layer, succesively building more abstract features.
When building networks with many layers, this idea is called deep learning.
Two special network architectures are important for the success of deep learning.
Convolutional networks~\cite{lecun1998gradient} process the image by looking repeatedly at small patches of the input.
This is good at finding locally restricted features, and is especially good for spatial data, such as images or videos.

Another important technique are recurrent neural networks.
TODO
Image classification, video classification
tagging sequence
RNN/LSTM
enters a sequence

While deep learning was popular in the last years, research in shot boundary detection, on the contrary, has ceased in the last years.
From 2001 until 2007, the TRECVid~\cite{trecvid} conference series was hosting tracks on shot boundary detection.
The TRECVID committee provided a video test collection with manually annotated gold data, which could be used by different research groups to develop and test their algorithms.
The tasks range from instance detection (e.g. detecting a person in a video), semantic indexing, event detection and many more.
However, as told before, the last challenge for shot boundary detection was in 2007.

We think this decline in research is largely because the current approaches are highly developed, and there is not much potential for further improvements.

This is why we propose to use deep learning approaches for shot boundary detection.
The next sections will detail our approach.

