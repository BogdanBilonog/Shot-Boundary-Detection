\subsection{Evaluation}
\label{sec:hard_cut_evaluation}

\begin{table}[ht]
	\centering
	\begin{tabular}{l|l}
	Precision & XX \% \\ \hline
	Recall & XX \% \\ \hline
	Accuracy & XX \% \\
	\end{tabular}
	\caption{Results of the hard cut detection}
	\label{tab:hard_cut_results}
\end{table}

\subsubsection{Training Data}
The SVM is trained on an mixture of videos from the 2007 TrecVid contest.
The videos in their full length do not make a good training set, as the ratio between hard cuts and other frames is about \texttt{1:1000}, leading to a failure to predict any hard-cuts.
A custom training set must be constructed, which contains a suffcient amount of hard cuts.
The choice of non-hard-cuts is also crucial for achieving good classification performance: too homogenous frames will lower the decision boundary, and lead to bad performance on noisy examples.
Therefore the negative set should also include soft-cuts and particularly noisy frames.
The training set we finally build consists of about 250 hard cuts and 2200 negatives.
